% !TeX root = ../libro.tex
% !TeX encoding = utf8

\chapter*{Glosario}
\addcontentsline{toc}{chapter}{Glosario} % Añade el glosario a la tabla de contenidos

\begin{description}
	\item[$\N$] Conjunto de números naturales con el $0$ incluido.
	
	\item[$\Z$] Conjunto de números enteros.
	
	\item[$\R$] Conjunto de números reales.
	
	\item[$\C$] Conjunto de números complejos.
	
	\item[$|C|$] Cantidad de elementos distintos en el conjunto $C$ o su cardinal.
	
	\item[$\sum$] Sumatoria de varios elementos.
	
	\item[$\prod$] Producto de varios elementos.
	
	\item[$\int_{a}^{b}$] Integral definida en el intervalo $[a, b]$.
	
	\item[$f'(x)$] Derivada de la función $f(x)$.
	
	\item[$(a, b)$] Máximo Común Divisor de $a$ y $b$.
	
	\item[${[a, b]}$] Mínimo Común Múltiplo de $a$ y $b$.
	
	\item[$LCM(m)$] Mínimo Común Múltiplo de los $m$ primeros números.
	
	\item[$res(a; b)$] Resto de dividir $a$ entre $b$.
	
	\item[$a\mid b$] El número $a$ divide a $b$.
	
	\item[$a\nmid b$] El número $a$ no divide a $b$.
	
	\item[$\phi(n)$] Función $\phi$ de Euler.
	
	\item[$\mathcal{U}(A)$] Grupo formado por las unidades del anillo $A$.
	
	\item[${A[x]}$] Anillo de polinomios con coeficientes en el anillo $A$.
	
	\item[$or(a)$] Orden del elemento $a$ de un grupo $G$, es decir, el menor $k$ tal que $a^k = 1$.
	
	\item[$ord_n(a)$] Orden de $a$ módulo $n$, es decir, el menor $k$ tal que $a^k \equiv 1 \mod(n)$.
	
	\item[$\Phi_n$] $n$-ésimo polinomio ciclotómico.
	
	\item[$\binom{n}{k}$] Binomio de $n$ sobre $k$.
	
	\item[$n!$] Producto de los primeros $n$ números o factorial de $n$.
	
	\item[$\left(\frac{a}{b}\right)$] Símbolo de Jacobi/Legendre de $a$ y $b$.
	
	\item[$a \equiv b \mod(n)$] Congruencia de $a$ con $b$ módulo $n$.
	
	\item[$\mathbb{F}_q$] Cuerpo finito de tamaño $q$ potencia de un primo.
	
	\item[$O(f(n))$] Complejidad asintótica máxima.
	
	\item[$\Omega(f(n))$] Complejidad asintótica mínima.
	
	\item[$\Theta(f(n))$] Complejidad asintótica exacta.
	
	\item[$O^\sim(f(n))$] Equivalente a $O(f(n)poly(\log(f(n)))$ donde $poly(n)$ es un polinomio.
	
	\item[$\lfloor x \rfloor$] Parte entera por deceso de $x$.
	
	\item[$\lceil x \rceil$] Parte entera por exceso de $x$.
	
	\item[$\log(n)$] Logaritmo en base $2$ de $n$.
	
	\item[$\ln(n)$] Logaritmo natural o en base $e$ de $n$.
	
	\item[AKS] Algoritmo desarrollado por \textit{Agrawal}, \textit{Kayal} y \textit{Saxena}.
	
	\item[FFT] Transformada Rápida de Fourier.
	
	\item[DFT] Transformada Discreta de Fourier.
	
	\item[GCC] GNU C Compiler.
	
	\item[Clang] LLVM C/C++ Lang Compiler.
	
	\item[MSVC] Microsoft Visual C and C++ Compiler.
	
	\item[CMake] Generador de sistemas de compilación (Meta Build System).
	
	\item[Conan] Manejador de paquetes de C y C++
	
	\item[C++] Lenguaje de Programación C++.
	
	\item[GMP] GNU Mutiprecision Library.
	
	\item[MPFR] GNU Multiprecision Floating-Point Reliable Library.
	
	\item[NTL] Number Theory Library.
\end{description}
\endinput
