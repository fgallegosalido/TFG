% !TeX root = ../libro.tex
% !TeX encoding = utf8

\chapter{Código Fuente}\label{codigo_fuente}

En este apéndice se van a explicar algunas cosas del código fuente que quedan fuera del trabajo principal. El código fuente se puede encontrar en \cite{fgallego_tfg_github}.

\section{Generación de Números Primos}

En la última sección del trabajo nos dedicamos a comparar el test \textbf{AKS} con los test probabilísticos de \textit{Miller-Rabin} y de \textit{Solovau-Strassen}.\\

Para dichas comparaciones era necesario generar distintos números primos que nos ayudarían a presentar resultados con distintos conjuntos y combinaciones. Para generar los números primos, se ha considerado que la mejor manera de hacerlo es usando la función que proporciona \textbf{GMP} llamada \textit{mpz\_probab\_prime\_p}. Esta función devuelve $2$ si el número es definitivamente primo, $1$ si es probablemente primo y $0$ si es definitivamente compuesto.\\

El conjunto de números primos generado son los mayores primos que ocupan $k$ bits. Por ejemplo, $13$ es el mayor primo que ocupa $4$ bits. Se han generado primos hasta los $512$ bits. El código para generarlos es el siguiente:

\begin{lstlisting}
auto biggestNBitsPrime(std::size_t bits) -> mpz_class {
	assert(bits >= 2);

	auto n = 2_mpz;
	mpz_pow_ui(n.get_mpz_t(), n.get_mpz_t(), bits);
	--n;

	while (true) {
		if (mpz_probab_prime_p(n.get_mpz_t(), 50) > 0) {
			return n;
		}
		n -= 2;
	}
}

auto generateTestPrimes(std::size_t n) -> std::vector<mpz_class> {
	auto primes = std::vector<mpz_class>(n);

	std::ranges::generate(primes, [bits = std::size_t(1)]() mutable {
		return biggestNBitsPrime(++bits);
	});

	return primes;
}
\end{lstlisting}

La función \textit{biggestNBitsPrime} recibe la cantidad de bits y devuelve el mayor primo que ocupa esa cantidad de bits.\\

La función \textit{generateTestPrimes} recibe hasta cuántos bits tiene que generar primos, y va generando el mayor primo para cada cantidad de bits hasta llegar a la indicada.\\

En el \textit{main} se llama tal que así:

\begin{lstlisting}
const auto primes = generateTestPrimes(512);
\end{lstlisting}

Tenemos así un vector con los primos descritos hasta $512$ bits. Luego se combinan convenientemente según la comparación que se esté haciendo.

\section{Generación de Gráficas}

Las gráficas generadas en este trabajo son la combinación de dos herramientas.\\

Por un lado generar los datos, lo cual se hace en C++ y se encuentra en el archivo \textbf{examples/src/Benchmarks.cpp}. Dicho archivo fuente usa la librería desarrollada para generar los tiempos.\\

Una vez generados los tiempos de ejecución, los datos se almacenan en distintos archivos \textbf{.txt}. Estos archivos luego se usan en un script de \textit{Gnuplot}, el cual se encarga de generar las imágenes que contienen las gráficas, y que han sido incluidas en este trabajo.\\

El comando más importante es el que nos permite pintar las gráficas. Dicho comando es \textit{plot}, al cual le podemos pasar distintos archivos de datos o funciones juntos con distintos parámetros (como el estilo de las gráficas y el título), y las guarda en un archivo \textbf{.png}.\\

Para adaptar las eficiencias teóricas a los datos que hemos generado se ha usado el comando \textit{fit}, el cual se encarga de adaptar los coeficientes para que sean más fieles a los datos.

\endinput
%------------------------------------------------------------------------------------
% FIN DEL APÉNDICE. 
%------------------------------------------------------------------------------------
