% !TeX root = ../libro.tex
% !TeX encoding = utf8
%
%*******************************************************
% Resumen
%*******************************************************

\chapter{Resumen}

Los números primos son de vital importancia en las matemáticas en general y, en particular, la \textit{Teoría de Números}. De especial interés son aquellas propiedades que nos permiten determinar si un número es primo. Dicho interés se basa en encontrar maneras rápidas y eficientes de probar dichas propiedades. Estos tests suelen usarse en ramas como la criptografía, que es la base de la seguridad en Internet. De ahí la importancia de estudiar los tests de primalidad.\\

En este trabajo vamos a presentar un test de primalidad que es general, determinista, polinómico e incondicional, llamado \textbf{AKS} en honor a sus tres autores. Además desarrollaremos una implementación y lo compararemos con otros tests probabilísticos usados en la actualidad, como pueden ser el test de \textit{Miller-Rabin} y el test de \textit{Solovay-Strassen}. Comprobaremos que el test \textbf{AKS} queda por detrás en cuanto a eficiencia, por lo que su utilidad práctica es baja.

\endinput