% !TeX root = ../libro.tex
% !TeX encoding = utf8
\chapter{Herramientas Matemáticas}

En este primer capítulo vamos a describir herramientas básicas del álgebra que nos van a servir para entender mejor los conceptos y demostraciones que presentaremos más adelante.\\

Empezaremos dando una introducción a distintos espacios de trabajo como los cuerpos, los grupos, los anillos, etc.\\

Después introduciremos los conceptos de combinatoria y álgebra modular junto con algunas propiedades que nos serán imprescindibles para presentar el trabajo de la manera más clara posible.\\

Haremos también una pequeña presentación de los polinomios ciclotómicos, los cuales son de vital importancia y nos serán muy útiles en la demostración del algoritmo \textbf{AKS}.\\

Haremos una breve introducción a la \textit{Hipótesis Generaliza de Riemann}, la cual nos servirá para ver resultados mejorados a los que presentaremos.\\

Finalmente haremos una introducción a la notación que usaremos para estudiar la complejidad algorítmica.

\section{Estructuras Algebraicas}

Para trabajar con muchos de los elementos que presentaremos a continuación, es necesario hacerlo bajo diversas estructuras matemáticas con ciertas propiedades.\\

Presentaremos las que más nos servirán en el desarrollo del trabajo.

\subsection{Anillos}

Sea $R$ un conjunto no vacío y sean dos aplicaciones $(+), (\cdot)$ definidas por

\begin{alignat*}{2}
	(+): R \times R & \to & R \\
	(a, b) & \mapsto & a+b, \\
	(\cdot): R \times R & \to & R \\
	(a, b) & \mapsto & ab,
\end{alignat*}

Dichas aplicaciones las llamaremos suma y producto respectivamente.

\begin{definicion}
	La tupla $(R, +, \cdot)$ es un anillo si cumple las siguientes propiedades:
	
	\begin{itemize}
		\item \textbf{Asociatividad de la suma}. Para todo $a, b, c \in R$, se cumple que $(a + b) + c = a + (b + c)$.
		
		\item \textbf{Conmutatividad de la suma}. Para todo $a, b \in R$, se cumple que $a + b = b + a$.
		
		\item \textbf{Elemento neutro para la suma}. Existe $e \in R$ tal que $a + e = a$ para todo $a \in R$. Dicho elemento se suele representar con el número cero, $0$.
		
		\item \textbf{Inverso para la suma}. Para todo $a \in R$ existe $b \in R$ tal que $a + b = 0$. Dicho elemento se suele conocer como el opuesto de $a$, y se representa con $-a$.
		
		\item \textbf{Asociatividad del producto}. Para todo $a, b, c \in R$, se cumple que $(ab)c = a(bc)$.
		
		\item \textbf{Elemento neutro para el producto}. Existe $e \in R$ tal que $ae = ea = a$ para todo $a \in R$. Dicho elemento se suele representar con el número uno, $1$.
		
		\item \textbf{Distributividad de la suma respecto del producto}. Para todo $a, b, c \in R$, se cumplen:
		
		\[ a(b + c) = ab + ac \]
		\[ (a + b)c = ac + bc \]
	\end{itemize}
	
	Además, se dice que $(R, +, \cdot)$ es conmutativo si cumple:
	
	\begin{itemize}
		\item \textbf{Conmutatividad del producto}. Para todo $a, b \in R$, se cumple que $ab = ba$.
	\end{itemize}
\end{definicion}

Vamos ahora a definir las unidades de un anillo.

\begin{definicion}
	Sea $A$ un anillo conmutativo. Diremos que $a \in A$ es una unidad si tiene inverso respecto del producto. Esto es que existe $b \in A$ tal que $ab = ba = 1$. Dicho elemento se conoce como el inverso de $a$, y se suele representar con $a^{-1}$.\\
	
	Al conjunto de las unidades de un anillo se le denota por $\mathcal{U}(A)$. Se dice que $a \in A$ es un divisor de cero si existe $b \in A \setminus \{0\}$ tal que $ab = 0$.
\end{definicion}

A la operación de sumar el opuesto se le suele llamar \textit{resta}, y se representa con el símbolo $-$ (es decir, $a + (-b) = a - b$). A la operación de multiplicar por el inverso se le suele llamar \textit{dividir}, y se representa con el símbolo $/$ (es decir, $ab^{-1} = a/b$).

Ahora vamos a dar algunas propiedades de los anillos.

\begin{proposicion}
	Sea $A$ un anillo. Se cumplen entonces:
	
	\begin{itemize}
		\item Los elementos neutros tanto para la suma como para el producto son únicos.
		
		\item El opuesto de cada elemento es único.
		
		\item Para todo $a \in A$, se cumple que $a0 = 0$.
		
		\item Para todo $a, b \in A$, se cumple que $(-a)b = -(ab) = a(-b)$.
		
		\item Sea $a \in \mathcal{U}(A)$, entonces su inverso es único.
		
		\item Sean $a, b \in \mathcal{U}(A)$, entonces $ab \in \mathcal{U}(A)$ y $(ab)^{-1} = b^{-1}a^{-1}$. Esta propiedad implica además que $\mathcal{U}(A)$ es un grupo, concepto que explicaremos más adelante.
	\end{itemize}
\end{proposicion}

Dadas estas propiedades de los anillos, vamos a pasar a definir dos estructuras con propiedades muchos más deseables.

\begin{definicion}
	Sea $A$ un anillo conmutativo. Diremos que $A$ es un \textit{dominio de integridad} si el elemento neutro para la suma, $0$, es el único divisor de cero.\\
	
	Esto implica que $A$ es un dominio de integridad si, y solo si, dados $a, b \in A$ con $ab = 0$, entonces se cumple que $a = 0$ ó $b = 0$.
\end{definicion}

\begin{definicion}
	Sea $A$ un anillo conmutativo. Diremos que $A$ es un \textit{cuerpo} si todo elemento no nulo es unidad, es decir, $\mathcal{U}(A) = A \setminus \{0\}$.\\
	
	Esto implica que $A$ es un cuerpo si, y solo si, todo elemento de $A \setminus \{0\}$ tiene inverso.
\end{definicion}

Ahora vamos a presentar algunos ejemplos de anillos.

\begin{ejemplo}
	$\Z$ es un anillo junto con las operaciones suma y producto clásicas.
\end{ejemplo}

\begin{ejemplo}
	Sea $n\Z$ el conjunto de los múltiplos de $n$ en $\Z$. Es un anillo con las operaciones heredadas de $\Z$.
\end{ejemplo}

\begin{ejemplo}
	Sea $\Z/n\Z$ una clase de equivalencia tal que dos elementos de $\Z$ son iguales si, y solo si, son múltiplos de $n$. A este conjunto se le suele denotar con $\Z_n$, y también es un anillo con las siguientes operaciones.
	
	\begin{alignat}{2}
	(+): \Z_n \times \Z_n & \to & \Z_n \\
	(a, b) & \mapsto & res(a+b; n), \\
	(\cdot): \Z_n \times \Z_n & \to & \Z_n \\
	(a, b) & \mapsto & res(ab; n),
	\end{alignat}
	
	donde $res(a; n)$ es el resto de dividir $a \in \Z$ entre $n$.\\
	
	Así pues, si tomamos por ejemplo $\Z_5$, entonces $6$ y $11$ son equivalentes en este anillo, pues el resto de dividirlos entre $5$ es el mismo.\\
	
	A estos anillos se les suele conocer como \textit{anillos modulares}.
\end{ejemplo}

\begin{ejemplo}
	Sea $A$ un anillo conmutativo, entonces el conjunto de los polinomios con coeficientes en $A$, $A[x]$, es también un anillo conmutativo.
\end{ejemplo}

\begin{ejemplo}
	Sea $A$ un anillo y $f \in A[x]$ un polinomio con coeficientes en $A$. Entonces la clase de equivalencia $A[x]/f(x)$ donde dos polinomios con coeficientes en $A$ son equivalentes si tienen el mismo resto al dividirlos por $f(x)$, es un anillo. En particular, si $A$ es un cuerpo y $f$ es irreducible en $A$, entonces $A[x]/f(x)$ es un cuerpo.\\
	
	Por ejemplo, $\Z_p/(x+1)$ con $p$ primo es un cuerpo.
\end{ejemplo}

\begin{ejemplo}
	$\Q$, $\R$ y $\C$ son ejemplos de cuerpos infinitos.
\end{ejemplo}

\begin{ejemplo}
	Todos los cuerpos finitos se denominan por $\mathbb{F}_q$ con $q = p^k$ con $p$ primo y $k \geq 1$. Además, dos cuerpos finitos con el mismo cardinal son equivalentes.
\end{ejemplo}

En el desarrollo del trabajo, usaremos sobre todo los anillos modulares y los anillos modulares de polinomios.\\

También haremos uso del conjunto de las unidades de los anillos modulares, es decir, $\mathcal{U}(\Z_n)$, a veces también notados como $\Z_n^*$ o $\Z_n^\times$.\\

Definido el conjunto de las unidades del anillo y sabiendo que $\Z_n$ es un anillo, es natural definir entonces la \textit{Función $\phi$ de Euler}.

\begin{definicion}\label{funcion_phi_de_euler}
	Sea $n > 1$. Se define la \textit{Función $\phi$ de Euler} como
	
	\begin{equation}
	\phi(n) = |\mathcal{U}(\Z_n)|,
	\end{equation}
	
	donde la operación $|\cdot|$ es el cardinal de un conjunto.
\end{definicion}

Una propiedad que nos será útil más adelante sobre los cuerpos es la relacionada con las raíces de los polinomios con coeficientes en un cuerpo. Enunciamos pues la siguiente proposición.

\begin{proposicion}\label{raices_en_cuerpos}
	Sea $f \in F[x]$ con $F$ un cuerpo. Entonces $f$ tiene a lo sumo tantas raíces distintas como el grado de $f$.
\end{proposicion}

\begin{proof}
	Haremos esta prueba por inducción  sobre el grado de $f$, siendo este $n$. Entonces:
	
	\begin{itemize}
		\item Sea $f(x) = a \neq 0$, entonces $f$ no tiene raíces. Del mismo modo, si $f(x) = ax + b$ con $a \neq 0$, entonces $-a^{-1}b$ es la única raíz de $f$.
		
		\item Supongamos ahora la proposición cierta para todos los polinomios de grado $n$ y sea $f(x) = a_0 + a_1x + \dotso + a_nx^n + a_{n+1}x^{n+1}$ con $a_{n+1} \neq 0$, luego de grado $n+1$. Si $f$ no tiene raíces en $F$, entonces la proposición se cumple trivialmente, así que supongamos que $f$ tiene al menos una raíz $c \in F$. Entonces tenemos que $\exists g \in F[x]$ tal que $f(x) = (x - c)g(x)$.\\
		
		Es entonces claro que el grado de $g$ es $n$, y por hipótesis de inducción tenemos que $g$ tiene como mucho $n$ raíces distintas, luego $f$ tiene como mucho $n+1$ raíces distintas.
	\end{itemize}
\end{proof}

\subsection{Grupos}

Sea $G$ un conjunto no vacío y sea $(\cdot)$ una operación interna en $G$ definida como

\begin{alignat*}{2}
	(\cdot): G \times G & \to & G \\
	(x, y) & \mapsto & xy,
\end{alignat*}

a la cual llamaremos producto. Damos entonces la siguiente definición.

\begin{definicion}
	La pareja $(G, \cdot)$ es un grupo si se cumplen las siguientes propiedades:
	
	\begin{itemize}
		\item \textbf{Asociatividad}. Para todo $x, y, z \in G$, se tiene que $(xy)z = x(yz)$.
		
		\item \textbf{Elemento neutro}. Existe $e \in G$ tal que $ex = xe = x$ para todo $x \in G$. Dicho elemento se suele representar con el número uno, $1$.
		
		\item \textbf{Inverso}. Para todo $x \in G$ existe $y \in G$ tal que $xy = yx = 1$. Dicho elemento se suele conocer como el inverso de $x$, y se representa con el símbolo $x^{-1}$.
	\end{itemize}
	
	Además, se dice que $(G, \cdot)$ es un grupo abeliano si cumple:
	
	\begin{itemize}
		\item \textbf{Conmutatividad}. Para todo $x, y \in G$, se tiene que $xy = yx$.
	\end{itemize}
\end{definicion}

Al cardinal del conjunto $G$ lo denominaremos \textit{orden del grupo $G$}, y lo representamos por $|G|$. En palabras más simples, se trata de la cantidad de elementos distintos que contiene el grupo $G$. Si $|G| < \infty$, entonces decimos que se trata de un \textit{grupo finito}.\\

Denominaremos \textit{orden} de un elemento al menor $k \geq 1$ tal que dicho elemento multiplicado $k$ veces es igual a $1$. Se denota por $\mathop{or}(a)$ con $a \in G$, donde $G$ es un grupo cualquiera.\\

Algunas propiedades inmediatas y fáciles de comprobar son las siguientes.

\begin{proposicion}
	Sea $(G, \cdot)$ un grupo. Entonces:
	
	\begin{itemize}
		\item El elemento neutro es único.
		
		\item Para cada $x \in G$, su inverso $x^{-1}$ es único.
		
		\item \textbf{Involución}. Para cada $x \in G$, $(x^{-1})^{-1} = x$.
		
		\item Si $xx = x$ con $x \in G$, entonces $x = 1$.
		
		\item \textbf{Cancelación}. Sean $x, y, z \in G$, entonces:
		
		\[ xy = xz \Rightarrow y = z \]
		\[ yx = zx \Rightarrow y = z \]
		
		\item El inverso del elemento neutro es él mismo.
		
		\item Para todo $x, y \in G$, se cumple que $(xy)^{-1} = y^{-1}x^{-1}$.
		
		\item Para todo $x, y \in G$, existen únicos $u, v \in G$ tales que:
		
		\[ xu = y \]
		\[ vx = y \]
	\end{itemize}
\end{proposicion}

Existen muchos ejemplos de grupos, como por ejemplo $\Z$, $\Q$, $\R$ y $\C$ bajo la operación de la suma.\\

Además de los anteriores, existen muchas categorías de grupos, entre los que podemos encontrar los grupos de permutaciones, los grupos diédricos, los cuaternios, etc. No vamos a centrarnos en ellos más, pues no los necesitaremos más adelante.\\

Sin embargo, y como ya dijimos anteriormente, dado un anillo conmutativo $A$, el conjunto de la unidades de dicho anillo, $\mathcal{U}(A)$, es un grupo. En especial, nos vamos a centrar en los anillos $\Z_n$ con $n > 1$ y sus correspondiente grupos de unidades, $\mathcal{U}(\Z_n) = \Z_n^*$, también conocido como grupo multiplicativo de $\Z_n$.

\section{Combinatoria}

En esta sección vamos a presentar algunos resultados en el campo de la combinatoria, los cuales serán útiles más adelante.\\

Empecemos por definir la operación del binomio, la cual aparece en la fórmula de los coeficientes del \textit{Binomio de Newton}.

\begin{definicion}
	Sean $n, k \in \Z$ con $n \geq k \geq 0$. Entonces definimos el binomio de la forma
	
	\begin{equation}\label{formula_binomio}
	\binom{n}{k} = \frac{n!}{k!(n - k)!},
	\end{equation}
	
	donde $n! = 1\cdot2\dotsm n$ es la operación factorial de $n$.
\end{definicion}

Existen muchas propiedades de los binomios, pero solo presentaremos algunas que utilizaremos en el desarrollo de la teoría.

\begin{proposicion}\label{propiedades_binomio}
	Se cumplen:
	
	\begin{enumerate}
		\item \[ \binom{n}{k} = \binom{n-1}{k} + \binom{n-1}{k-1}\;\;\forall n \geq k > 0 \]
		
		\item \[ \binom{n}{k} = \binom{n}{n-k}\;\;\forall n \geq k \geq 0 \]
		
		\item \textbf{Identidad del Palo de Hockey}. Sean $n, k$ tales que $n \geq k \geq 0$. Entonces
		
		\begin{equation}\label{identidad_del_palo_de_hockey}
		\sum_{i=k}^{n}\binom{i}{k} = \binom{n+1}{k+1}
		\end{equation}
	\end{enumerate}
\end{proposicion}

Ahora vamos a presentar el \textit{Binomio de Newton}, propiedad que nos vendrá muy bien en algunas demostraciones.

\begin{teorema}{(Binomio de Newton)}\label{binomio_de_newton}
	Sean $x, y \in \Z$ (a nosotros nos vale con $\Z$, pero $x$ e $y$ pueden pertenecer a otros espacios más generales), y sea $n \in \Z$ no negativo. Entonces se cumple
	
	\[ (x + y)^n = \sum_{k=0}^{n}\binom{n}{k}x^ky^{n-k} \]
\end{teorema}

La demostración se puede hacer por inducción sobre $n$, por lo que no la vamos a detallar.\\

Ahora veremos algunos resultados que usaremos más adelante.

\begin{lema}\label{binomio_cota_inferior_2n}
	\begin{equation}
	\binom{2n + 1}{n} > 2^{n+1}\;\;\;\;\forall n \geq 2
	\end{equation}
\end{lema}

\begin{proof}
	Haremos una inducción sobre $n$. Sea entonces pues $n=2$, y tenemos
	
	\[ \binom{2\cdot2 + 1}{2} = \binom{5}{2} = \frac{5!}{2!3!} = 10 > 8 = 2^{2+1} \]
	
	Supuesto cierto para $n$, comprobemos la desigualdad para $n+1$:
	
	\[ \binom{2(n+1) + 1}{n+1} = \frac{(2n+3)!}{(n+1)!(n+2)!} = 2\frac{(2n+3)}{(n+2)}\binom{2n+1}{n} > \frac{(2n+3)}{(n+2)}2^{n+2} > 2^{n+2} \]
	
	En la penúltima desigualdad hemos aplicado la hipótesis de inducción sobre $n$, y la última desigualdad se deduce de que $\frac{(2n+3)}{(n+2)} = 2 - \frac{1}{n+2} > 1$.
\end{proof}

\section{Máximo Común Divisor y Mínimo Común Múltiplo}

En esta sección vamos a introducir un concepto que es una de las bases de la \textit{Teoría de Números} y del \textit{Álgebra} en general.\\

Damos entonces la siguiente definición.

\begin{definicion}
	Sean $a, b \in \Z$. Diremos que $d \in \Z$ con $d \geq 0$ es el \textit{Máximo Común Divisor} de $a$ y $b$ si se cumplen:
	
	\begin{itemize}
		\item $d \mid a$ y $d \mid b$.
		
		\item Sea $d' \in \Z$ tal que $d' \mid a$ y $d' \mid b$, entonces $d' \mid d$.
	\end{itemize} 
	
	Ha dicho $d$ se le suele denotar por $\mathop{mcd}(n)$ (siglas en español), $\gcd(a, b)$ (siglas del nombre en inglés, \textit{Greatest Common Divisor}) ó $(a, b)$. Esta última suele ser la más utilizada por no depender del idioma, y será la que usaremos a lo largo del trabajo.
\end{definicion}

Como veremos más adelante en el \textit{Algoritmo de Euclides} \ref{algoritmo_de_euclides}, necesitamos saber qué ocurre cuando uno de los dos parámetros es $0$. Dado entonces $a \in \Z$, se tiene que $(a, 0) = (0, a) = |a|$ por la propia definición. Además, por convención, se tiene que $(0, 0) = 0$.\\

En esencia, el \textit{Máximo Común Divisor} se puede entender como el mayor número entero positivo de manera que divide a dos números dados. Existe también el concepto opuesto, llamado \textit{Mínimo Común Múltiplo}, que es la contraparte del \textit{Máximo Común Divisor}, el cual podemos definir de la siguiente manera.

\begin{definicion}
	Sean $a, b \in \Z$. Diremos que $d \in \Z$ con $d \geq 0$ es el \textit{Mínimo Común Múltiplo} de $a$ y $b$ si se cumplen:
	
	\begin{itemize}
		\item $a \mid d$ y $b \mid d$.
		
		\item Sea $d' \in \Z$ tal que $a \mid d'$ y $b \mid d'$, entonces $d \mid d'$.
	\end{itemize} 
	
	Ha dicho $d$ se le suele denotar por $\mathop{mcm}(n)$ (siglas en español), $lcm(a, b)$ (siglas del nombre en inglés, \textit{Least Common Multiplier}) ó $[a, b]$. Esta última suele ser la más utilizada por no depender del idioma, y será la que usaremos a lo largo del trabajo.
\end{definicion}

La conexión entre ambos conceptos viene dada por el siguiente teorema.

\begin{teorema}\label{}
	Sean $a, b \in \Z$, entonces se cumple
	
	\begin{equation}
	(a, b)[a, b] = |ab|
	\end{equation}
\end{teorema}

Del mismo modo que pasa con el \textit{Máximo Común Divisor}, el $0$ se comporta mal con esta definición, luego por convención se denota $[a, 0] = [0, a] = [0, 0] = 0$ para todo $a \in \Z$.\\

En esencia es el menor entero de manera que es múltiplo tanto de ambos números.\\

Vamos a dar ahora algunas propiedades del \textit{Máximo Común Divisor}.

\begin{proposicion}\label{propiedades_maximo_comun_divisor}
	Sean $a, b \in \Z$. Entonces se tiene:
	
	\begin{enumerate}
		\item \textbf{Existencia y Unicidad}. Existe un único $d \in \Z$ tal que $(a, b) = d$.
		
		\item \textbf{Identidad de Bézout}. Existen $x, y \in \Z$ tales que $ax + by = (a, b)$.
		
		\item $(ca, cb) = c(a, b)$ para todo $c > 0$.
		
		\item Sea $c > 0$ tal que $c \mid a$ y $c \mid b$. Entonces $\left(\frac{a}{c}, \frac{b}{c}\right) = \frac{(a, b)}{c}$.
		
		\item $(a, b) = (b, a) = (a, -b) = (a, b + ac)$ para todo $c \in \Z$.
		
		\item Para todo $c \in \Z$ tal que $b \mid ac$, se tiene que $b \mid (a, b)c$.
		
		\item Si $(a, b) = 1$ y $b \mid ac$ para algún $c \in \Z$, entonces $b \mid c$.
		
		\item Si $b \mid a$ y sea $c \in \Z$ tal que $c \mid a$ y $(b, c) = 1$, entonces se tiene que $bc \mid a$.
		
		\item Sea $c \in \Z$, entonces $(a, bc) = 1$ si, y solo si, $(a, b) = (a, c) = 1$.
	\end{enumerate}
\end{proposicion}

Una manera elemental de calcular el \textit{Máximo Común Divisor} de dos números es calcular los factores primos de ambos, seleccionar aquellos que coincidan y multiplicarlos. Este método es fácil de enseñar y aplicar. Sin embargo, a medida que las entradas se hacen más grandes, calcular los factores primos puede convertirse en una tarea difícil. Es por ello que \textit{Euclides} desarrolló un algoritmo que acelera este proceso, conocido como \textit{Algoritmo de Euclides}. Antes de mostrar el algoritmo, vamos a dar un par de propiedades que prueban su validez.

\begin{proposicion}
	Sean $a, b \in \Z$. Se tienen:
	
	\item Si $a, b \neq 0$ y $b \mid a$, entonces $(a, b) = |b|$.
	
	\item Si $a = qb + r$ para algunos $q, r \in \Z$, entonces $(a, b) = (b, r)$.
\end{proposicion}

Estas dos propiedades nos permiten mostrar un algoritmo para calcular el \textit{Máximo Común Divisor} que no requiere de calcular los factores primos de ambos números, y solo se basa en los restos de la división de enteros. Mostramos entonces el algoritmo a continuación.

\begin{algorithm}[H]
	\caption{Algoritmo de Euclides}\label{algoritmo_de_euclides}
	\begin{algorithmic}[1]
		\Procedure{GCD}{$a$, $b$}\Comment{Calcula el \textit{máximo común divisor} de $a$ y $b$}
			\If{$a = b = 0$}
				\State \Return{$0$}
			\EndIf
			\If{$a = 0$}
				\State \Return{$|a|$}
			\EndIf
			\If{$b = 0$}
				\State \Return{$|b|$}
			\EndIf
			\State \Return{GCD($b$, $res(a; b)$)}
		\EndProcedure
	\end{algorithmic}
\end{algorithm}

Este algoritmo junto con la definición de \textit{Máximo Común Divisor} será fundamental en todo el trabajo.

\section{Aritmética Modular}

En este apartado nos vamos a centrar en la aritmética modular tanto con enteros como con polinomios. La mayoría de propiedades son las mismas, y solo distinguiremos entre ambos cuando sea necesario. En general, nos referiremos a aritmética de enteros, pero era necesario aclarar que dichas propiedades serán equivalentes para polinomios (por tratarse ambos de anillos conmutativos).\\

Empecemos por definir lo que es una congruencia.

\begin{definicion}
	Sean $a, b \in \Z$ y $n \in \N \setminus \{0\}$. Diremos que $a$ y $b$ son \textit{congruentes módulo $n$} si el resto de dividir ambos por $n$ es el mismo.\\
	
	Esto lo denotaremos por $a \equiv b \pmod{n}$, $a \equiv b \mod(n)$ ó $a \equiv_n b$. De la propia definición se sobreentiende que $b \equiv a \mod(n)$.
\end{definicion}

Es importante destacar que esta operación es una relación de equivalencia:

\begin{itemize}
	\item \textbf{Reflexividad}. $a \equiv a \mod(n)$ para todos $a, n$.
	
	\item \textbf{Simetría}. Se cumple $a \equiv b \mod(n)$ y $b \equiv a \mod(n)$ para todos $a, b, n$.
	
	\item \textbf{Transitividad}. Si $a \equiv b \mod(n)$ y $b \equiv c \mod(n)$, entonces $a \equiv c \mod(n)$ para todos $a, b, c, n$.
\end{itemize}

De la definición podemos deducir varias propiedades inmediatas.

\begin{proposicion}
	Sea $n \in \N \setminus \{0\}$. Se cumplen entonces:
	
	\begin{enumerate}
		\item Si $a \equiv b \mod(n)$, entonces $a + k \equiv b + k \mod(n)$ para todo $k$.
		
		\item Si $a \equiv b \mod(n)$, entonces $ka \equiv kb \mod(n)$ para todo $k$.
		
		\item Si $a \equiv b \mod(n)$ y $c \equiv d \mod(n)$, entonces $a + c \equiv b + d \mod(n)$.
		
		\item Si $a \equiv b \mod(n)$ y $c \equiv d \mod(n)$, entonces $a - c \equiv b - d \mod(n)$.
		
		\item Si $a \equiv b \mod(n)$ y $c \equiv d \mod(n)$, entonces $ac \equiv bd \mod(n)$.
		
		\item Si $a \equiv b \mod(n)$, entonces $a^k \equiv b^k \mod(n)$ para todo $k$.
		
		\item Si $a \equiv b \mod(n)$ y $p \in \Z[x]$, entonces $p(a) \equiv p(b) \mod(n)$.
		
		\item Si $a \equiv b \mod(\phi(n))$ y sea $k$ tal que $(k, n) = 1$, entonces $k^a \equiv k^b \mod(n)$.
				
		\item Si $a + k \equiv b + k \mod(n)$ para algún $k$, entonces $a \equiv b \mod(n)$.
		
		\item Si $ka \equiv kb \mod(n)$ para algún $k$ tal que $(k, n) = 1$, entonces $a \equiv b \mod(n)$.
		
		\item Si $ka \equiv kb \mod(kn)$ para algún $k$, entonces $a \equiv b \mod(n)$.
		
		\item Existe un único $a^{-1}$ tal que $aa^{-1} \equiv 1 \mod(n)$ si, y solo si, $(a, n) = 1$. A $a^{-1}$ se le llama el \textit{inverso multiplicativo de $a$ módulo $n$}
		
		\item Si $a \equiv b \mod(n)$ y $(a, n) = (b, n) = 1$, entonces $a^{-1} \equiv b^{-1} \mod(n)$.
		
		\item Si $ax \equiv b \mod(n)$ con $(a, n) = 1$, entonces $x \equiv a^{-1}b \mod(n)$ es solución de la ecuación.
	\end{enumerate}
\end{proposicion}

\begin{proof}
	Vamos a demostrar algunas de estas propiedades, pues muchas de ellas son casi triviales.
	
	\begin{itemize}
		\item[6] Si $a \equiv b \mod(n)$, entonces $a = qn + b$ para algún $q \in \Z$. Elevamos a $k$ y nos queda por \autoref{binomio_de_newton} lo siguiente
		
		\[ a^k = (qn + b)^k = \sum_{i=0}^{k}\binom{k}{i}q^in^ib^{k-i} = b^k + \sum_{i=1}^{k}\binom{k}{i}q^in^ib^{k-i} = b^k + n\sum_{i=0}^{k-1}\binom{k}{i}q^{i+1}n^ib^{k-i-1} \]
		
		Llamando $q' = \sum_{i=0}^{k-1}\binom{k}{i}q^{i+1}n^ib^{k-i-1}$, tenemos que $a^k = q'n + b^k$, luego $a^k \equiv b^k \mod(n)$.
		
		\item[7] Se deduce de las propiedades $1$, $2$ y $6$.
		
		\item[8] Si $a \equiv b \mod(\phi(n))$, entonces $a = q\phi(n) + b$ para algún $q \in \Z$, luego nos queda
		
		\begin{equation}
		k^a = k^{q\phi(n) + b} = \left(k^{\phi(n)}\right)^qk^b \equiv k^b \mod(n)
		\end{equation} 
		
		En la última equivalencia hemos usado la propiedad $2$ y \autoref{teorema_de_euler}, ya que $(k, n) = 1$.
		
		\item[12] Vamos a probar ambas implicaciones.
		
		\begin{itemize}
			\item[$\Rightarrow$] Si $aa^{-1} \equiv 1 \mod(n)$, entonces $aa^{-1} + n(-q) = 1$ para algún $q \in \Z$. Puesto que $ax+ny = (a, n)$ para algunos $x, y \in \Z$ por la \textit{Identidad de Bézout} \ref{propiedades_maximo_comun_divisor}, entonces podemos asegurar que $(a, n) = 1$.
			
			\item[$\Leftarrow$] Como $(a, n) = 1$, por la \textit{Identidad de Bézout} \ref{propiedades_maximo_comun_divisor} tenemos que $\exists x,y \in \Z$ tales que $ax + ny = 1 \Rightarrow ax \equiv 1 \mod(n)$, luego hemos probado la existencia.\\
			
			Supongamos ahora que existe un $b \not\equiv a^{-1} \mod(n)$ tal que $ab \equiv 1 \mod(n)$. Se tiene entonces
			
			\begin{align}
			b &\equiv b(aa^{-1}) \mod(n)\\
			&\equiv (ba)a^{-1} \mod(n)\\
			&\equiv a^{-1} \mod(n)
			\end{align}
			
			Llegamos así a una contradicción, luego $a^{-1}$ es único.
		\end{itemize}
	\end{itemize}
\end{proof}

Presentaremos ahora una propiedad interesante de las congruencias.

\begin{lema}\label{lema_abp_apbp}
	Para todo $a, b \in \Z$ y para todo $p$ primo, se tiene que $(a + b)^p \equiv a^p + b^p \mod(p)$
\end{lema}

\begin{proof}
	Por un lado, sabemos que $(a + b)^p = \sum_{i=0}^{p}\binom{p}{i}a^ib^{p-i}$.\\
	
	Sabiendo eso, consideremos los binomios dentro de la sumatoria, pero excluyendo los casos donde $i = 0$ e $i = p$:
	
	\[ \binom{p}{i} = \frac{p!}{i!(p - i)!} \]
	
	Como $p$ es primo, entonces $p \nmid k!$ para todo $0 < k < p$ ó, lo que es lo mismo, $k!$ no contiene el número $p$ en su factorización. Como $0 < i < p$ y, en consecuencia, $0 < p - i < p$, tenemos que ni $i!$ ni $(p - i)!$ contienen en su factorización a $p$, y por lo tanto no lo contiene el producto.\\
	
	Como el binomio es un número entero, tenemos entonces que $\binom{p}{i}$ contiene en su factorización a $p$ o, lo que es lo mismo, que es múltiplo de $p$. Esto último implica que, para $0 < i < p$, tenemos
	
	\[ \binom{p}{i}a^ib^{p-i} \equiv 0 \mod(p) \]
	
	Así tenemos que
	
	\[ (a + b)^p = \sum_{i=0}^{p}\binom{p}{i}a^ib^{p-i} \equiv a^p + b^p \mod(p) \]
\end{proof}

\begin{teorema}{(Pequeño Teorema de Fermat)}\label{pequenio_teorema_de_fermat}
	Sean $n \geq 0$ y $p$ primo. Entonces se cumple que $n^p \equiv n \mod(p)$.
\end{teorema}

\begin{proof}
	Procederemos usando inducción sobre $n$.\\
	
	Para el caso $n = 0$ tenemos que $0^p \equiv 0 \mod(p)$, que es trivialmente cierto.\\
	
	Aplicamos ahora inducción y suponemos que se cumple para $n$, por lo que vamos a comprobarlo para $n + 1$.
	
	\[ (n + 1)^p \equiv n^p + 1^p \mod(p) \]
	
	Usando la hipótesis de inducción sobre $n$ y que $1^p = 1$, tenemos
	
	\[ (n + 1)^p \equiv n + 1 \mod(p) \]
	
	Es justo lo que queríamos probar.
\end{proof}

Del teorema que acabamos de demostrar, es evidente comprobar que, dado $n \in \Z$, entonces $n^{p-1} \equiv 1 \mod(p)$ para todo $p$ primo. Este hecho nos da una pista de una generalización del \textit{Pequeño Teorema de Fermat}, la cual fue descubierta por \textit{Euler}.

\begin{teorema}{(Teorema de Euler)}\label{teorema_de_euler}
	Sean $n, p > 1$ con $n$ y $p$ coprimos, es decir, $(n, p) = 1$. Entonces se cumple que $n^{\phi(p)} \equiv 1 \mod(p)$, siendo $\phi$ la función de Euler.
\end{teorema}

Aquí podemos ver que el \textit{Pequeño Teorema de Fermat} es un caso particular de este teorema, pues $\phi(p) = p-1$ si, y solo si, $p$ es primo.\\

Vamos a dar ahora una definición que está íntimamente relacionada con estos resultados y que será de vital importancia más adelante.

\begin{definicion}
	Sean $n > 1$ y $a \in \Z$ tales que $(a, n) = 1$. Se define el \textit{orden de $a$ módulo $n$}, como el menor $k > 0$ tal que $a^k \equiv 1 \mod(n)$. Dicho $k$ se denota por $ord_n(a)$
\end{definicion}

Por el \textit{Teorema de Euler}, podemos deducir claramente que $ord_n(a) \mid \phi(n)$ para todo $a \in \Z$ tal que $(a, n) = 1$.

\section{Polinomios Ciclotómicos}

Lo primero que vamos a hacer es dar una propiedad de las raíces de los polinomios del estilo $x^n - a \in \C[x]$ con $a \in \C$ y $n \geq 1$.

\begin{lema}
	Sea $a \in \C$ y $n \geq 1$. Entonces el polinomio $x^n - a \in \C[x]$ tiene exactamente $n$ raíces distintas
\end{lema}

\begin{proof}
	Si $n = 1$, entonces $a$ es la única raíz, luego supongamos $n > 1$.\\
	
	Supongamos ahora que existe $r \in \C$ raíz doble de $x^n - a$. Entonces debe existir un $g \in \C[x]$ tal que $x^n - a = (x - r)^2g(x)$. Derivando ambas partes obtenemos lo siguiente
	
	\[ nx^{n-1} = (x - r)(2g(x) + (x - r)g'(x)) \]
	
	Luego tanto $x^n - a$ como $nx^{n-1}$ son divisibles por $x - r$, luego $x - r \mid (x^n - a, nx^{n-1})$. Pero $(x^n - a, nx^{n-1}) = a$ y $x - r \nmid a$, lo cual es una contradicción y por lo tanto $r$ no puede ser una raíz doble, luego todas las raíces de $x^n - a$ son distintas.
\end{proof}

A estos $n$ números complejos (raíces de $x^n - a$) vamos a llamarlos \textbf{raíces $n$-ésimas} de $a$. Si $n=2$, les llamamos \textbf{raíces cuadradas}, o \textbf{raíces cúbicas} si $n=3$. Si $a=1$, se les llama \textbf{raíces $n$-ésimas de la unidad}.\\

Para cada $n \geq 1$, dichas raíces conforman un subgrupo, $\C_n$, del grupo multiplicativo de los complejos, $\C^\times$, definido tal que:

\[ \C_n = \left\lbrace \zeta \in \C^\times : \zeta^n = 1 \right\rbrace = \left\lbrace \cos\left(\frac{2k\pi}{n}\right) + i\;\sin\left(\frac{2k\pi}{n}\right): k=0,...,n-1  \right\rbrace \]

Entre estas raíces, $\zeta_n = \cos\left(\frac{2\pi}{n}\right) + i\;\sin\left(\frac{2\pi}{n}\right)$ es llamada la \textbf{raíz $n$-ésima primitiva de la unidad}. A partir de aquí, es evidente comprobar que $\C_n = \left\langle \zeta_n \right\rangle$, lo cual lo hace un grupo cíclico de orden $n$ generado por $\zeta_n$.\\

Por otro lado, $\zeta_n^k$ es un generador de $\C_n$, o lo que es lo mismo, $or(\zeta_n^k) = n$ si, y solo si, $(n, k) = 1$. Por lo tanto definimos el conjunto de los generadores de $\C_n$ como:

\[ Gen(\C_n) = \left\lbrace \zeta \in \C_n : or(\zeta) = n \right\rbrace = \left\lbrace \zeta_n^k : 1 \leq k \leq n, (n, k) = 1 \right\rbrace \]

Es evidente ver que $\C_n$ tiene $\phi(n)$ generadores. Hacemos entonces la siguiente definición.

\begin{definicion}
	Sea $n \geq 1$, se define el $n$-ésimo polinomio ciclotómico $\Phi_n$ de la siguiente manera:
	
	\begin{equation}\label{formula_polinomio_ciclotomico}
	\Phi_n(x) = \prod_{\zeta \in Gen(\C_n)}(x - \zeta) = \prod_{\substack{1 \leq k \leq n \\ (n, k) = 1}}(x - \zeta_n^k)
	\end{equation}
\end{definicion}

Dicho de otro modo, $\Phi_n$ es el polinomio mónico de grado $\phi(n)$ donde las raíces $n$-ésimas de la unidad son de orden $n$. Ahora vamos a pasar a dar algunas propiedades de estos polinomios:

\begin{proposicion}
	Los $n$-ésimos polinomios ciclotómicos cumplen las siguientes propiedades:
	
	\begin{enumerate}
		\item $\Phi_n \in \Z[x]$
		
		\item $\Phi$ es irreducible en $\Q[x]$
		
		\item $x^n - 1 = \prod_{d|n}\Phi_d(x)$. En particular, $\Phi_n$ es el polinomio irreducible en $\Z[x]$ de mayor grado que divide a $x^n - 1$ y no divide a $x^k - 1$ con $1 \leq k < n$.
		
		\item Si restringimos los coeficientes de $\Phi_n$ a $\Z_p$ con $p$ primo y tal que $(p, n) = 1$, tenemos que $\Phi_n$ se puede factorizar en $\frac{\phi(n)}{d}$ polinomios irreducibles de grado $d$, donde $d = ord_n(p)$.
	\end{enumerate}
\end{proposicion}

Únicamente probaremos la cuarta propiedad por ser de mayor interés en el desarrollo del trabajo.

\begin{proof}
	Sea $\zeta$ una raíz de $\Phi_n$, y por tanto raíz $n$-ésima de la unidad. Como $(p^k, n) = 1$ para todo $k \geq 1$, entonces $\zeta^{p^k}$ es una raíz $n$-ésima de la unidad. Puesto que todos los automorfismos en un cuerpo finito son del tipo $\sigma(x) = x^{p^k}$ para algún $k$ \cite{mullen_panario_2013}, tenemos que son válidos, pues llevan raíces de la unidad en raíces de la unidad.\\
	
	Como $p^d \equiv 1 \mod(n)$, entonces el automorfismo $\sigma(x) = x^{p^d}$ es la identidad, y por tanto solo existen $d$ automorfismos distintos. Esto significa que para todas las raíces de $\Phi_n$, sus polinomios irreducibles tienen grado $d$. Dichos polinomios deben dividir además a $\Phi_n$ por estar compuestos por raíces de $\Phi_n$. Como el grado de $\Phi_n$ es $\phi(n)$ y cada polinomio irreducible tiene grado $d$, nos queda que $\Phi_n$ se puede factorizar en $\frac{\phi(n)}{d}$ polinomios de grado $d$.
\end{proof}

\section{Hipótesis Generalizada de Riemann}

En la rama del \textit{Análisis Matemático} y, en específico, la rama del \textit{Análisis en Variable Compleja}, existe una conjetura muy importante propuesta por \textit{Bernhard Riemann} en $1859$, cuya popularidad es debida a su inclusión entre uno de los Problemas del Milenio por el \textit{Clay Mathematics Institute}. Para enunciar dicha conjetura, definamos primero la \textit{Función Zeta de Riemann}, $\zeta(s)$, con $s \in \C$:

\begin{equation}\label{funcion_zeta_de_riemann}
\zeta(s) = \sum_{n=1}^{\infty}\frac{1}{n^s}
\end{equation}

Esta función se sabe que converge cuando la parte real de $s$ es mayor que $1$. Para los casos en los que la parte real de $s$ sea menor o igual que $1$, lo que se hace es extender analíticamente la función $\zeta$ de la siguiente manera:

\begin{equation}\label{funcion_zeta_de_riemann_extendida}
\zeta(s) = 2^s\pi^{s-1}\sin\left(\frac{\pi s}{2}\right)\Gamma(1-s)\zeta(1-s)
\end{equation}

Esta función está definida en todo $\C \setminus \{1\}$ (en $s = 1$ hay lo que se conoce como un \textit{polo} o \textit{singularidad}). La función $\Gamma$ extiende el concepto de factorial al plano complejo. Se define de la siguiente manera para $s$ con parte real positiva:

\begin{equation}
\Gamma(s) = \int_{0}^{\infty}t^{s-1}e^{-t}\mathop{dt}
\end{equation}

Como podemos ver en la propia definición de $\zeta$, dicha función se anula para todos los enteros negativos pares. Estos ceros son más conocidos como los \textit{ceros triviales} de la función $\zeta$. Existen también valores de $s$ cuya parte real se encuentra entre $0$ y $1$ (no incluidos) tales que $\zeta(s)$ también se anula. Estos valores son conocidos como los \textit{ceros no triviales} de la función $\zeta$.\\

Armados con este conocimiento, pasamos a enunciar la conjetura, también conocida como \textit{Hipótesis de Riemann}.

\begin{conjetura}\label{hipotesis_de_riemann}
	Todos los ceros no triviales de la función $\zeta$ tienen parte real igual a $\frac{1}{2}$.
\end{conjetura}

Esta conjetura, de ser cierta, implicaría profundos resultados en el ámbito de los números primos. En específico, existe una generalización de dicha conjetura, también denominada \textit{Hipótesis Generalizada de Riemann}, la cual se enuncia para un conjunto específico de funciones llamado \textit{Funciones-L de Dirichlet} y los \textit{Caracteres de Dirichlet}, los cuales no vamos a definir en este trabajo. El enunciado de la conjetura es el siguiente.

\begin{conjetura}\label{hipotesis_generalizada_de_riemann}
	Sea $\chi$ un \textit{Carácter de Dirichlet}. Se define $L$ como una función L de Dirichlet para todo $s \in \C \setminus \{1\}$ de la siguiente forma:
	
	\begin{equation}
	L(\chi, s) = \sum_{n=1}^{\infty}\frac{\chi(n)}{n^s}
	\end{equation}
	
	Entonces, si $L(\chi, s) = 0$ y la parte real de $s$ está entre $0$ y $1$ (no incluidos), la parte real de $s$ es igual a $\frac{1}{2}$.
\end{conjetura}

Es evidente comprobar que si tomamos $\chi(n) = 1$, tenemos la \textit{Hipótesis de Riemann} \ref{hipotesis_de_riemann}.\\

Más adelante mencionaremos esta conjetura, cuya veracidad implicaría mejoras en la complejidad del algoritmo \textbf{AKS}.

\section{Complejidad Algorítmica}

Para poder estudiar la complejidad algorítmica del test \textbf{AKS}, tenemos que entender qué es la complejidad algorítmica como tal. Para ello usaremos la notación asintótica $O$, $\Omega$ y $\Theta$.\\

Estas tres notaciones nos sirven para dar forma al concepto de crecimiento asintótico de una función.\\

Además, estas notaciones nos van a servir también para dar forma a la idea intuitiva de que el único término necesario en el comportamiento asintótico es aquel que crece más rápido.

\subsection{Notaciones de Complejidad Asintótica}

Para poder hablar de complejidad asintótica, usaremos distintas notaciones según el tipo de acotación asintótica que se está realizando.

\subsubsection{Notación $O$}

Empezaremos con el concepto intuitivo de que una función domina asintóticamente a otra según la entrada crece. Para ello damos la siguiente definición.

\begin{definicion}
	Sean $f$ y $g$ dos funciones definidas en $\N$, y cuyas imágenes pertenecen a $\R^+$. Diremos que $f$ es de orden $g$, notado como $O(g(n))$, si, y solo si, $\exists k \in \N$ y $\exists C \in \R^+$ tales que se cumple lo siguiente:
	
	$$f(n) \leq Cg(n) \;\;\;\forall n \in \N;\; n \geq k$$
\end{definicion}

Esta definición nos dice que una función domina a otra dada si la primera multiplicada por una constante es mayor que la segunda para toda entrada a partir de cierto punto. Veamos ahora algunos ejemplos:

\begin{ejemplo}
	Probar que $f(n) = 3n^2 + 1$ es $O(n^2)$.\\
	
	Tomando $k=1$ y $C=4$, podemos ver fácilmente usando inducción sobre $n$ que $3n^2 + 1 \leq 4n^2\;\forall n \geq 1$, luego podemos asegurar que $3n^2 + 1 = O(n^2)$.\\
	
	\begin{itemize}
		\item Si $n=1$, entonces $3 \cdot 1^2 + 1 = 4 \leq 4$, luego se cumple el caso inicial.
		\item Suponiendo cierto para $n$, comprobemos para $n + 1$. Entonces $3(n+1)^2 + 1 = 3n^2 + 6^n + 3 + 1 \leq 4n^2 + 6n + 3 \leq 4n^2 + 8n + 4 = 4(n+1)^2$, luego hemos probado lo que queríamos.
	\end{itemize}
\end{ejemplo}

\subsubsection{Notación $\Omega$}

Intuitivamente, el concepto de la notación $\Omega$ es el opuesto al concepto de la notación $O$. Lo vemos más rápido en la definición.

\begin{definicion}
	Sean $f$ y $g$ dos funciones definidas en $\N$, y cuyas imágenes pertenecen a $\R^+$. Diremos que $f$ es de orden $g$, notado como $\Omega(g(n))$, si, y solo si, $\exists k \in \N$ y $\exists C \in \R^+$ tales que se cumple lo siguiente:
	
	$$f(n) \geq Cg(n) \;\;\;\forall n \in \N;\; n \geq k$$
\end{definicion}

Viendo la definición, es inmediato ver que, dadas dos funciones $f,g:\N \to \R^+$, entonces $f(n) = O(g(n)) \Leftrightarrow g(n) = \Omega(f(n))$. Algunos ejemplos son:

\begin{ejemplo}
	$3^n = \Omega(2^n)$
\end{ejemplo}

\begin{ejemplo}
	$n^3 + 2n + 3 \neq \Omega(n^4)$
\end{ejemplo}

Realmente, este concepto es exactamente igual que el anterior, solo que la acotación la hacemos por debajo en vez de por arriba.

\subsubsection{Notación $\Theta$}

Este concepto no es más que una manera de indicar que dos funciones se acotan asintóticamente, o lo que es lo mismo, que crecen con la misma rapidez. También se le conoce como el ``orden exacto''. Para ser más exactos, esta es la definición.

\begin{definicion}
	Sean $f$ y $g$ dos funciones definidas en $\N$, y cuyas imágenes pertenecen a $\R^+$. Diremos que $f$ es de orden exacto $g$, notado como $\Theta(g(n))$, si, y solo si
	
	$$f(n) = O(g(n))\;\wedge\;f(n) = \Omega(g(n))$$
\end{definicion}

\subsubsection{Notación $O^\sim$}

Algunas veces es complicado calcular la complejidad exacta, y puede que nos baste simplemente probar que nuestro algoritmo pertenece a una clase que sigue siendo polinómica. Por ello hacemos la siguiente definición:

\begin{definicion}
	Sea $f:\N \to \R^+$ y definimos $O^\sim(f(n)) = O(f(n) \cdot poly(\log(f(n)))$, donde $poly(n)$ es una función polinómica en $n$.
\end{definicion}

Con esta definición, tenemos que $O^\sim(\log^k(n)) = O(\log^k(n) \cdot poly(\log(\log^k(n))) = O(\log^{k+\epsilon}(n))$.\\

Consideramos de ahora en adelante $\log(n)$ como el logaritmo en base $2$ de $n$, y $\ln(n)$ como el logaritmo natural (o en base $e$) de $n$.

\subsection{Propiedades}

Las propiedades que vamos a presentar ahora son comunes a todas las notaciones que hemos explicado anteriormente, por lo que usaremos $O(f(n))$ para referirnos a todas ellas.

\begin{proposicion}
	Sean $f, g$ dos funciones. Entonces se cumplen:
	
	\begin{enumerate}
		\item $O(f(n) + g(n)) = O(\max\{f(n), g(n)\})$, donde $\max\{f(n), g(n)\}$ es la función que crece asintóticamente más rápido.
		
		\item $O(f(n)g(n)) = O((f\cdot g)(n))$, donde $(f\cdot g)$ es la función producto.
		
		\item Sea $c > 0$ constante, entonces $O(cf(n)) = O(f(n))$.
		
		\item $O(\log_a(n)) = O(\log_b(n))$ para todos $a, b$ mayores que $1$.
	\end{enumerate}
\end{proposicion}

\endinput
