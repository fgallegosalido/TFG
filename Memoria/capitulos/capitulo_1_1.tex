% !TeX root = ../libro.tex
% !TeX encoding = utf8
\chapter{Herramientas Matemáticas}

En este primer capítulo vamos a describir herramientas básicas del álgebra que nos van a servir para entender mejor los conceptos y demostraciones que presentaremos más adelante.\\

Empezaremos dando una introducción a distintos espacios de trabajo como los cuerpos, los grupos, los anillos, etc.\\

Después introduciremos el concepto de álgebra modular junto con algunas propiedades que nos serán imprescindibles para presentar el trabajo de la manera más clara posible.\\

Haremos también una pequeña presentación de los polinomios ciclotómicos, los cuales son de vital importancia y nos serán muy útiles en la demostración del algoritmo \textbf{AKS}.

\section{Estructuras Algebraicas}

Para trabajar con muchos de los elementos que presentaremos a continuación, es necesario hacerlo bajo diversas estructuras matemáticas con ciertas propiedades.\\

Presentaremos las que más nos servirán en el desarrollo del trabajo.

\subsection{Anillos}

Sea $R$ un conjunto no vacío y sean dos aplicaciones $(+), (\cdot)$ definidas por

\begin{alignat*}{2}
	(+): R \times R & \to & R \\
	(a, b) & \mapsto & a+b, \\
	(\cdot): R \times R & \to & R \\
	(a, b) & \mapsto & ab,
\end{alignat*}

Dichas aplicaciones las llamaremos suma y producto respectivamente.

\begin{definicion}
	La tupla $(R, +, \cdot)$ es un anillo si cumple las siguientes propiedades:
	
	\begin{itemize}
		\item \textbf{Asociatividad de la suma}. Para todo $a, b, c \in R$, se cumple que $(a + b) + c = a + (b + c)$.
		
		\item \textbf{Conmutatividad de la suma}. Para todo $a, b \in R$, se cumple que $a + b = b + a$.
		
		\item \textbf{Elemento neutro para la suma}. Existe $e \in R$ tal que $a + e = a$ para todo $a \in R$. Dicho elemento se suele representar con el número cero, $0$.
		
		\item \textbf{Inverso para la suma}. Para todo $a \in R$ existe $b \in R$ tal que $a + b = 0$. Dicho elemento se suele conocer como el opuesto de $a$, y se representa con $-a$.
		
		\item \textbf{Asociatividad del producto}. Para todo $a, b, c \in R$, se cumple que $(ab)c = a(bc)$.
		
		\item \textbf{Elemento neutro para el producto}. Existe $e \in R$ tal que $ae = ea = a$ para todo $a \in R$. Dicho elemento se suele representar con el número uno, $1$.
		
		\item \textbf{Distributividad de la suma respecto del producto}. Para todo $a, b, c \in R$, se cumplen:
		
		\[ a(b + c) = ab + ac \]
		\[ (a + b)c = ac + bc \]
	\end{itemize}
	
	Además, se dice que $(R, +, \cdot)$ es conmutativo si cumple:
	
	\begin{itemize}
		\item \textbf{Conmutatividad del producto}. Para todo $a, b \in R$, se cumple que $ab = ba$.
	\end{itemize}
\end{definicion}

Vamos ahora a definir las unidades de un anillo.

\begin{definicion}
	Sea $A$ un anillo conmutativo. Diremos que $a \in A$ es una unidad si tiene inverso respecto del producto. Esto es que existe $b \in A$ tal que $ab = ba = 1$ (dicho elemento se conoce como el inverso de $a$, y se suele representar con $a^{-1}$).\\
	
	Al conjunto de las unidades de un anillo se le denota por $\mathcal{U}(A)$. Se dice que $a \in A$ es un divisor de cero si existe $b \in A \setminus \{0\}$ tal que $ab = 0$.
\end{definicion}

A la operación de sumar el opuesto podemos llamarla \textit{resta}, y se representa con el símbolo $-$ (es decir, $a + (-b) = a - b$. A la operación de multiplicar por el inverso podemos llamarla \textit{dividir}, y se representa con el símbolo $/$ (es decir, $ab^{-1} = a/b$).

Ahora vamos a dar algunas propiedades de los anillos.

\begin{proposicion}
	Sea $A$ un anillo. Se cumplen entonces:
	
	\begin{itemize}
		\item Los elementos neutros tanto para la suma como para el producto son únicos.
		
		\item El opuesto de cada elemento es único.
		
		\item Para todo $a \in A$, se cumple que $a0 = 0$.
		
		\item Para todo $a, b \in A$, se cumple que $(-a)b = -(ab) = a(-b)$.
		
		\item Sea $a \in \mathcal{U}(A)$, entonces su inverso es único.
		
		\item Sean $a, b \in \mathcal{U}(A)$, entonces $ab \in \mathcal{U}(A)$ y $(ab)^{-1} = b^{-1}a^{-1}$. Esta propiedad implica además que $\mathcal{U}(A)$ es un grupo, concepto que explicaremos más adelante.
	\end{itemize}
\end{proposicion}

Dadas estas propiedades de los anillos, vamos a pasar a definir dos estructuras con propiedades muchos más deseables.

\begin{definicion}
	Sea $A$ un anillo conmutativo. Diremos que $A$ es un \textit{dominio de integridad} si el elemento neutro para la suma, $0$, es el único divisor de cero.\\
	
	Esto implica que $A$ es un dominio de integridad si, y solo si, dados $a, b \in A$ con $ab = 0$, entonces se cumple que $a = 0$ ó $b = 0$.
\end{definicion}

\begin{definicion}
	Sea $A$ un anillo conmutativo. Diremos que $A$ es un \textit{cuerpo} si todo elemento no nulo es unidad, es decir, $\mathcal{U}(A) = A \setminus \{0\}$.\\
	
	Esto implica que $A$ es un cuerpo si, y solo si, todo elemento de $A \setminus \{0\}$ tiene inverso.
\end{definicion}

Ahora vamos a presentar algunos ejemplos de anillos.

\begin{ejemplo}
	$\Z$, $\Z_n$ y $A[x]$ ($A$ siendo anillo conmutativo y $n > 1$) son ejemplos de anillos conmutativos.
\end{ejemplo}

\begin{ejemplo}
	$\Q$, $\R$, $\C$ y $\mathbb{F}_p$ ($p$ potencia de un primo) son ejemplos de cuerpos, siendo $\mathbb{F}_p$ los únicos finitos. 
\end{ejemplo}

Definido el conjunto de las unidades del anillo y sabiendo que $\Z_n$ es un anillo, es natural definir entonces la \textit{Función $\phi$ de Euler}.

\begin{definicion}\label{funcion_phi_de_euler}
	Sea $n > 1$. Se define la \textit{Función $\phi$ de Euler} como
	
	\begin{equation}
	\phi(n) = |\mathcal{U}(\Z_n)|
	\end{equation}
\end{definicion}

Una propiedad que nos será útil más adelante sobre los cuerpos es la relacionada con las raíces de los polinomios con coeficientes en un cuerpo. Enunciamos pues la siguiente proposición.

\begin{proposicion}\label{raices_en_cuerpos}
	Sea $f \in F[x]$ con $F$ un cuerpo. Entonces $f$ tiene a lo mucho tantas raíces distintas como el grado de $f$.
\end{proposicion}

\begin{proof}
	Haremos esta prueba por inducción  sobre el grado de $f$, siendo este $n$. Entonces:
	
	\begin{itemize}
		\item Sea $f(x) = a \neq 0$, entonces $f$ no tiene raíces. Del mismo modo, si $f(x) = ax + b$ con $a \neq 0$, entonces $-a^{-1}b$ es la única raíz de $f$.
		
		\item Supongamos ahora la proposición cierta para todos los polinomios de grado $n$ y sea $f(x) = a_0 + a_1x + \dotso + a_nx^n + a_{n+1}x^{n+1}$ con $a_{n+1} \neq 0$, luego de grado $n+1$. Si $f$ no tiene raíces en $F$, entonces la proposición se cumple trivialmente, así que supongamos que $f$ tiene al menos una raíz $c \in F$. Entonces tenemos que $\exists g \in F[x]$ tal que $f(x) = (x - c)g(x)$.\\
		
		Es entonces claro que el grado de $g$ es $n$, y por hipótesis de inducción tenemos que $g$ tiene como mucho $n$ raíces distintas, luego $f$ tiene como mucho $n+1$ raíces distintas.
	\end{itemize}
\end{proof}

En el desarrollo del trabajo, usaremos sobre todo $\Z_n$ con $n > 1$, también conocidos como anillos modulares. Es importante destacar que, si $n$ es primo, entonces $\Z_n$ es un cuerpo.\\

También haremos uso del conjunto de las unidades de dichos anillos, es decir, $\mathcal{U}(\Z_n)$, a veces también notados como $\Z_n^*$ o $\Z_n^\times$.\\

$\Z_n$ podemos entenderlo también como una clase de equivalencia, donde dos elementos $a, b \in \Z$ son equivalentes si, y solo si, los restos de dividir $a$ y $b$ son los mismos. Por ejemplo, $6$ y $11$ son equivalente en $\Z_5$, pues el resto de dividir ambos por $5$ es $1$. Con esta definición, muchas veces se denota este anillo con $\Z / n\Z$, donde $n\Z$ corresponde al anillo de los múltiplos de $n$. Por conveniencia, seguiremos utilizando $\Z_n$ para notar estos anillos.\\

De la misma manera que presentamos $\Z_n$, también presentamos los anillos modulares de polinomios. Por ejemplo, en $\Z_n[x] / (x^2 - 1)$ (ó $(\Z/n\Z)[x]/(x^2 - 1)$) con $n > 1$, tenemos polinomios con coeficientes en $\Z_n$, donde dos polinomios son equivalentes si, y solo si, el resto de dividir ambos por $x^2 - 1$ es el mismo. Estos anillos los utilizaremos extensivamente en la demostración del algoritmo \textbf{AKS}, y por eso es necesario presentarlos aquí.

\subsection{Grupos}

Sea $G$ un conjunto no vacío y sea $(\cdot)$ una operación interna en $G$ definida como

\begin{alignat*}{2}
	(\cdot): G \times G & \to & G \\
	(x, y) & \mapsto & xy,
\end{alignat*}

a la cual llamaremos producto. Damos entonces la siguiente definición.

\begin{definicion}
	La pareja $(G, \cdot)$ es un grupo si se cumplen las siguientes propiedades:
	
	\begin{itemize}
		\item \textbf{Asociatividad}. Para todo $x, y, z \in G$, se tiene que $(xy)z = x(yz)$.
		
		\item \textbf{Elemento neutro}. Existe $e \in G$ tal que $ex = xe = x, \forall x \in G$. Dicho elemento se suele representar con el número uno, $1$.
		
		\item \textbf{Inverso}. Para todo $x \in G$ existe $y \in G$ tal que $xy = yx = 1$. Dicho elemento se suele conocer como el inverso de $x$, y se representa con el símbolo $x^{-1}$.
	\end{itemize}
	
	Además, se dice que $(G, \cdot)$ es un grupo abeliano si cumple:
	
	\begin{itemize}
		\item \textbf{Conmutatividad}. Para todo $x, y \in G$, se tiene que $xy = yx$.
	\end{itemize}
\end{definicion}

Al cardinal del conjunto $G$ lo denominaremos \textit{orden del grupo $G$}, y lo representamos por $|G|$. En palabras más simples, se trata de la cantidad de elementos distintos que contiene el grupo. Si $|G| < \infty$, entonces decimos que se trata de un \textit{grupo finito}.\\

Algunas propiedades inmediatas y fáciles de comprobar son las siguientes.

\begin{proposicion}
	Sea $(G, \cdot)$ un grupo. Entonces:
	
	\begin{itemize}
		\item El elemento neutro es único.
		
		\item Para cada $x \in G$, su inverso $x^{-1}$ es único.
		
		\item \textbf{Involución}. Para cada $x \in G$, $(x^{-1})^{-1} = x$.
		
		\item Si $xx = x$ con $x \in G$, entonces $x = 1$.
		
		\item \textbf{Cancelación}. Sean $x, y, z \in G$, entonces:
		
		\[ xy = xz \Rightarrow y = z \]
		\[ yx = zx \Rightarrow y = z \]
		
		\item El inverso del elemento neutro es él mismo.
		
		\item Para todo $x, y \in G$, se cumple que $(xy)^{-1} = y^{-1}x^{-1}$.
		
		\item Para todo $x, y \in G$, existen únicos $u, v \in G$ tales que:
		
		\[ xu = y \]
		\[ vx = y \]
	\end{itemize}
\end{proposicion}

Existen muchos ejemplos de grupos, como por ejemplo $\Z$, $\Q$, $\R$ y $\C$ bajo la operación de la suma.\\

Además de los anteriores, existen muchas categorías grupos, entre los que podemos encontrar los grupos de permutaciones, los grupos diédricos, los cuaternios, etc. No vamos a centrarnos en ellos más, pues no los necesitaremos más adelante.\\

Sin embargo, y como ya dijimos anteriormente, dado un anillo conmutativo $A$, el conjunto de la unidades de dicho anillo, $\mathcal{U}(A)$, es un grupo. En especial, nos vamos a centrar en los anillos $\Z_n$ con $n > 1$ y sus correspondiente grupos de unidades, $\mathcal{U}(\Z_n) = \Z_n^*$, también conocido como grupo multiplicativo de $\Z_n$.\\

Es importante destacar lo siguiente:

\begin{proposicion}
	Sea $n \in \N$ con $n > 1$. Entonces $|\Z_n^*| = \phi(n)$, donde $\phi$ es la función de Euler.
\end{proposicion}

En la siguiente sección nos dedicaremos a introducirnos en los conceptos de aritmética modular más en profundidad.

\section{Combinatoria}

En esta sección vamos a presentar algunos resultados en el campo de la combinatoria, los cuales serán útiles más adelante.\\

Empecemos por definir la operación del binomio, la cual aparece en la fórmula de los coeficientes del binomio de Newton.

\begin{definicion}
	Sean $n, k \in \Z$ con $n \geq k \geq 0$. Entonces definimos el binomio de la forma
	
	\[ \binom{n}{k} = \frac{n!}{k!(n - k)!} \],
	
	Donde $n!$ es la operación factorial de $n$.
\end{definicion}

Existen muchas propiedades de los binomios, pero solo presentaremos algunas que utilizaremos en el desarrollo de la teoría.

\begin{proposicion}
	Se cumplen:
	
	\begin{enumerate}
		\item \[ \binom{n}{k} = \binom{n-1}{k} + \binom{n-1}{k-1}\;\;\forall n \geq k > 0 \]
		
		\item \[ \binom{n}{k} = \binom{n}{n-k}\;\;\forall n \geq k \geq 0 \]
		
		\item \textbf{Identidad del Palo de Hockey}. Sean $n, k$ tales que $n \geq k \geq 0$. Entonces
		
		\begin{equation}\label{identidad_del_palo_de_hockey}
		\sum_{i=k}^{n}\binom{i}{k} = \binom{n+1}{k+1}
		\end{equation}
	\end{enumerate}
\end{proposicion}

Ahora veremos algunos resultados que usaremos más adelante.

\begin{lema}\label{binomio_cota_inferior_2n}
	\[ \binom{2n + 1}{n} > 2^{n+1}\;\;\;\;\forall n \geq 2 \]
\end{lema}

\begin{proof}
	Haremos una inducción sobre $n$. Sea entonces pues $n=2$, y tenemos
	
	\[ \binom{2\cdot2 + 1}{2} = \binom{5}{2} = \frac{5!}{2!3!} = 10 > 8 = 2^{2+1} \]
	
	Supuesto cierto para $n$, comprobemos la desigualdad para $n+1$:
	
	\[ \binom{2(n+1) + 1}{n+1} = \frac{(2n+3)!}{(n+1)!(n+2)!} = 2\frac{(2n+3)}{(n+2)}\binom{2n+1}{n} > \frac{(2n+3)}{(n+2)}2^{n+2} > 2^{n+2} \]
	
	En la penúltima desigualdad hemos aplicado la hipótesis de inducción sobre $n$, y la última desigualdad se deduce de que $\frac{(2n+3)}{(n+2)} = 2 - \frac{1}{n+2} > 1$.
\end{proof}

\section{Aritmética Modular}

En este apartado nos vamos a centrar en la aritmética modular tanto con enteros como con polinomios. La mayoría de propiedades son las mismas, y solo distinguiremos entre ambos cuando sea necesario. En general, nos referiremos a aritmética de enteros, pero era necesario aclarar que dichas propiedades serán equivalente para polinomios (por tratarse ambos de anillos conmutativos).\\

Empecemos por definir lo que es una congruencia.

\begin{definicion}
	Sean $a, b \in \Z$ y $n \in \N \setminus \{0\}$. Diremos que $a$ y $b$ son \textit{congruentes módulo $n$} si el resto de dividir ambos por $n$ es el mismo.\\
	
	Esto lo denotaremos por $a \equiv b \pmod{n}$, $a \equiv b \mod(n)$ ó $a \equiv_n b$. De la propia definición se sobreentiende que $b \equiv a \mod(n)$.
\end{definicion}

Es importante destacar que esta operación es una relación de equivalencia:

\begin{itemize}
	\item \textbf{Reflexividad}. $a \equiv a \mod(n)$ para todos $a, n$.
	
	\item \textbf{Simetría}. Se cumple $a \equiv b \mod(n)$ y $b \equiv a \mod(n)$ para todos $a, b, n$.
	
	\item \textbf{Transitividad}. Si $a \equiv b \mod(n)$ y $b \equiv c \mod(n)$, entonces $a \equiv c \mod(n)$ para todos $a, b, c, n$.
\end{itemize}

De la definición podemos deducir varias propiedades inmediatas.

\begin{proposicion}
	Sea $n \in \N \setminus \{0\}$. Se cumplen entonces:
	
	\begin{enumerate}
		\item Si $a \equiv b \mod(n)$, entonces $a + k \equiv b + k \mod(n)$ para todo $k$.
		
		\item Si $a \equiv b \mod(n)$, entonces $ka \equiv kb \mod(n)$ para todo $k$
		
		\item Si $a \equiv b \mod(n)$ y $c \equiv d \mod(n)$, entonces $a + c \equiv b + d \mod(n)$.
		
		\item Si $a \equiv b \mod(n)$ y $c \equiv d \mod(n)$, entonces $a - c \equiv b - d \mod(n)$.
		
		\item Si $a \equiv b \mod(n)$ y $c \equiv d \mod(n)$, entonces $ac \equiv bd \mod(n)$.
		
		\item Si $a \equiv b \mod(n)$, entonces $a^k \equiv b^k \mod(n)$ para todo $k$
		
		\item Si $a \equiv b \mod(n)$ y $p \in \Z[x]$, entonces $p(a) \equiv p(b) \mod(n)$.
		
		\item Si $a \equiv b \mod(\phi(n))$, entonces $k^a \equiv k^b \mod(n)$ para algún $k$ tal que $(k, n) = 1$.
				
		\item Si $a + k \equiv b + k \mod(n)$ para algún $k$, entonces $a \equiv b \mod(n)$.
		
		\item Si $ka \equiv kb \mod(n)$ para algún $k$ tal que $(k, n) = 1$, entonces $a \equiv b \mod(n)$.
		
		\item Si $ka \equiv kb \mod(kn)$ para algún $k$, entonces $a \equiv b \mod(n)$.
		
		\item Existe un único $a^{-1}$ tal que $aa^{-1} \equiv 1 \mod(n)$ si, y solo si, $(a, n) = 1$. A $a^{-1}$ se le llama el \textit{inverso multiplicativo de $a$ módulo $n$}
		
		\item Si $a \equiv b \mod(n)$ y $(a, n) = (b, n) = 1$, entonces $a^{-1} \equiv b{-1} \mod(n)$.
		
		\item Si $ax \equiv b \mod(n)$ con $(a, n) = 1$, entonces $x \equiv a^{-1}b \mod(n)$ es solución de la ecuación.
	\end{enumerate}
\end{proposicion}

Ahora vamos a presentar el \textit{Binomio de Newton}, propiedad que nos vendrá muy bien en algunas demostraciones.

\begin{teorema}{(Binomio de Newton)}\label{binomio_de_newton}
	Sean $x, y \in \Z$ (a nosotros nos vale con $\Z$, pero $x$ e $y$ pueden pertenecer a otros espacios más generales), y sea $n \in \Z$ no negativo. Entonces se cumple
	
	\[ (x + y)^n = \sum_{k=0}^{n}\binom{n}{k}x^ky^{n-k} \]
\end{teorema}

La demostración se puede hacer por inducción sobre $n$, por lo que no la vamos a detallar.\\

Presentaremos ahora una propiedad interesante de las congruencias.

\begin{lema}
	Para todo $a, b \in \Z$ y para todo $p$ primo, se tiene que $(a + b)^p \equiv a^p + b^p \mod(p)$
\end{lema}

\begin{proof}
	Por un lado, sabemos que $(a + b)^p = \sum_{i=0}^{p}\binom{p}{i}a^ib^{p-i}$.\\
	
	Sabiendo eso, consideremos los binomios dentro de la sumatoria, pero excluyendo los casos donde $i = 0$ e $i = p$:
	
	\[ \binom{p}{i} = \frac{p!}{i!(p - i)!} \]
	
	Como $p$ es primo, entonces $p \nmid k!$ para todo $0 < k < p$ ó, lo que es lo mismo, $k!$ no contiene el número $p$ en su factorización. Como $0 < i < p$ y, en consecuencia, $0 < p - i < p$, tenemos que ni $i!$ ni $(p - i)!$ contienen en su factorización a $p$, y por lo tanto no lo contiene el producto.\\
	
	Como el binomio es un número entero, tenemos entonces que $\binom{p}{i}$ contiene en su factorización a $p$ o, lo que es lo mismo, que es múltiplo de $p$. Esto último implica que, para $0 < i < p$, tenemos
	
	\[ \binom{p}{i}a^ib^{p-i} \equiv 0 \mod(p) \]
	
	Así tenemos que
	
	\[ (a + b)^p = \sum_{i=0}^{p}\binom{p}{i}a^ib^{p-i} \equiv a^p + b^p \mod(p) \]
\end{proof}

\begin{teorema}{(Pequeño Teorema de Fermat)}\label{pequenio_teorema_de_fermat}
	Sean $n > 1$ y $p$ primo. Entonces se cumple que $n^p \equiv n \mod(p)$.
\end{teorema}

\begin{proof}
	Procederemos usando inducción sobre $n$.\\
	
	Para el caso $n = 0$ tenemos que $0^p \equiv 0 \mod(p)$, que es trivialmente cierto.\\
	
	Aplicamos ahora inducción y suponemos que se cumple para $n$, por lo que vamos a comprobarlo para $n + 1$.
	
	\[ (n + 1)^p \equiv n^p + 1^p \mod(p) \]
	
	Usando la hipótesis de inducción sobre $n$ y que $1^p = 1$, tenemos
	
	\[ (n + 1)^p \equiv n + 1 \mod(p) \]
	
	Es justo lo que queríamos probar.
\end{proof}

Del teorema que acabamos de demostrar, es evidente comprobar que, dado $n \in \Z$, entonces $n^{p-1} \equiv 1 \mod(p)$ para todo $p$ primo. Este hecho nos da una pista de una generalización del pequeño teorema de Fermat, la cual fue descubierta por Euler.

\begin{teorema}{(Teorema de Euler)}
	Sean $n, p > 1$ con $n$ y $p$ coprimos, es decir, $(n, p) = 1$. Entonces se cumple que $n^{\phi(p)} \equiv 1 \mod(p)$, siendo $\phi$ la función de Euler.
\end{teorema}

Aquí podemos ver que el Pequeño Teorema de Fermat es un caso particular de este teorema, pues $\phi(p) = p-1$ si, y solo si, $p$ es primo.

\section{Polinomios Ciclotómicos}

Sea $a \in \C$ no nulo. El polinomio $x^n - a \in \C[x]$ con $n \geq 1$ tiene exactamente $n$ raíces distintas, pues la derivada de $x^n - a$ es $nx^{n-1}$, y $x$ (el cual es irreducible) no divide a $x^n - a$.\\

A estos $n$ números complejos (raíces de $x^n - a$) vamos a llamarlos \textbf{raíces $n$-ésimas} de $a$. Si $n=2$, les llamamos \textbf{raíces cuadradas}, o \textbf{raíces cúbicas} si $n=3$. Si $a=1$, se les llama \textbf{raíces $n$-ésimas de la unidad}.\\

Para cada $n \geq 1$, dichas raíces conforman un subgrupo, $\C_n$, del grupo multiplicativo de los complejos, $\C^\times$, definido tal que:

\[ \C_n = \left\lbrace \zeta \in \C^\times : \zeta^n = 1 \right\rbrace = \left\lbrace cos\left(\frac{2k\pi}{n}\right) + i\;sen\left(\frac{2k\pi}{n}\right): k=0,...,n-1  \right\rbrace \]

Entre estas raíces, $\zeta_n = cos\left(\frac{2\pi}{n}\right) + i\;sen\left(\frac{2\pi}{n}\right)$ es llamada la \textbf{raíz $n$-ésima primitiva de la unidad}. A partir de aquí, es evidente comprobar que $\C_n = \left\langle \zeta_n \right\rangle$, lo cual lo hace un grupo cíclico de orden $n$ generado por $\zeta_n$.\\

Por otro lado, $\zeta_n^k$ es un generador de $\C_n$, o lo que es lo mismo, $or(\zeta_n^k) = n$, si y solo si $(n, k) = 1$. Por lo tanto definimos el conjunto de los generadores de $\C_n$ como:

\[ Gen(\C_n) = \left\lbrace \zeta \in \C_n : or(\zeta) = n \right\rbrace = \left\lbrace \zeta_n^k : 1 \leq k \leq n, (n, k) = 1 \right\rbrace \]

Es evidente ver que $\C_n$ tiene $\phi(n)$ ($\phi$ es la función de Euler) generadores. Hacemos entonces la siguiente definición.

\begin{definicion}
	Sea $n \geq 1$, se define el $n$-ésimo polinomio ciclotómico, $\Phi_n$ tal que:
	
	\[ \Phi(x) = \prod_{\zeta \in Gen(\C_n)}(x - \zeta) = \prod_{\substack{1 \leq k \leq n \\ (n, k) = 1}}(x - \zeta_n^k) \]
\end{definicion}

Dicho de otro modo, $\Phi_n$ es el polinomio mónico de grado $\phi(n)$ donde las raíces $n$-ésimas de la unidad son de orden $n$. Ahora vamos a pasar a dar algunas propiedades de estos polinomios:

\begin{proposicion}
	Los $n$-ésimos polinomios ciclotómicos cumplen las siguientes propiedades:
	
	\begin{itemize}
		\item $\Phi_n \in \Z[x]$
		\item $\Phi$ es irreducible en $\Q[x]$
		\item $x^n - 1 = \prod_{d|n}\Phi_d(x)$. En particular, $\Phi_n$ es el polinomio irreducible en $\Z[x]$ de mayor grado que divide a $x^n - 1$ y no divide a $x^k - 1$ con $1 \leq k < n$.
		\item Si restringimos los coeficientes de $\Phi_n$ a $\Z_p$ con $p$ primo, y tal que $p \nmid n$, tenemos que $\Phi_n$ se puede factorizar en $\frac{\phi(n)}{d}$ polinomios irreducibles de grado $d$, donde $d = ord_n(p)$.
	\end{itemize}
\end{proposicion}

\section{Hipótesis Generalizada de Riemann}

En la rama del Análisis Matemático y, en específico, la rama del Análisis en Variable Compleja, existe una conjetura muy importante propuesta por Riemann, cuya popularidad es debida a su inclusión entre uno de los Problemas del Milenio por el Clay Mathematics Institute. Para enunciar dicha conjetura, definamos primero la \textit{Función Zeta de Riemann}, $\zeta(s)$, con $s \in \C$:

\begin{equation}\label{funcion_zeta_de_riemann}
\zeta(s) = \sum_{n=1}^{\infty}\frac{1}{n^s}
\end{equation}

Esta función se sabe que converge cuando la parte real de $s$ es mayor que $1$. Para los casos en los que la parte real de $s$ sea menor o igual que $1$, lo que se hace es extender analíticamente la función $\zeta$ de la siguiente manera:

\begin{equation}\label{funcion_zeta_de_riemann_extendida}
\zeta(s) = 2^s\pi^{s-1}\sin\left(\frac{\pi s}{2}\right)\Gamma(1-s)\zeta(1-s)
\end{equation}

Esta función está definida en todo $\C \setminus \{1\}$ (en $s = 1$ hay lo que se conoce como un \textit{polo}). La función $\Gamma$ extiende el concepto de factorial al plano complejo. Se define de la siguiente manera para $s$ con parte real positiva:

\begin{equation}
\Gamma(s) = \int_{0}^{\infty}t^{s-1}e^{-t}\mathop{dt}
\end{equation}

Como podemos ver en la propia definición de $\zeta$, dicha función se anula para todos los enteros negativos pares. Estos ceros son más conocidos como los \textit{ceros triviales} de la función $\zeta$. Existen también valores de $s$ cuya parte real se encuentra entre $0$ y $1$ (no incluidos) tales que $\zeta(s)$ también se anula. Estos valores son conocidos como los \textit{ceros no triviales} de la función $\zeta$.\\

Armados con este conocimiento, pasamos a enunciar la conjetura, también conocida como \textit{Hipótesis de Riemann}.

\begin{conjetura}\label{hipotesis_de_riemann}
	Todos los ceros no triviales de la función $\zeta$ tienen parte real igual a $\frac{1}{2}$.
\end{conjetura}

Esta conjetura, de ser cierta, implicaría profundos resultados en el ámbito de los números primos. En específico, existe una generalización de dicha conjetura, también denominada \textit{Hipótesis Generalizada de Riemann}, la cual se enuncia para un conjunto específico de funciones llamado \textit{Funciones-L de Dirichlet} y los \textit{Caracteres de Dirichlet}, los cuales no vamos a definir en este trabajo. El enunciado de la conjetura es el siguiente.

\begin{conjetura}\label{hipotesis_generalizada_de_riemann}
	Sea $\chi$ un \textit{Carácter de Dirichlet}. Se define $L$ como una función-L de Dirichlet para todo $s \in \C \setminus \{1\}$ de la siguiente forma:
	
	\begin{equation}
	L(\chi, s) = \sum_{n=1}^{\infty}\frac{\chi(n)}{n^s}
	\end{equation}
	
	Entonces, si $L(\chi, s) = 0$ y la parte real de $s$ está entre $0$ y $1$ (no incluidos), la parte real de $s$ es igual a $\frac{1}{2}$.
\end{conjetura}

Es evidente comprobar que si tomamos $\chi(n) = 1$, tenemos la \textit{Hipótesis de Riemann} \ref{hipotesis_de_riemann}.\\

Más adelante mencionaremos esta conjetura, cuya veracidad implicaría mejoras en la complejidad del algoritmo \textbf{AKS}.

\section{Complejidad Algorítmica}

Para poder estudiar la complejidad algorítmica del test AKS, tenemos que entender qué es la complejidad algorítmica como tal. Para ello usaremos la notación asintótica $O$, $\Omega$ y $\Theta$.\\

Estas tres notaciones nos sirven para dar forma al concepto de crecimiento asintótico de una función.\\

Además, estas notaciones nos van a servir también para dar forma a la idea intuitiva de que el único término necesario en el comportamiento asintótico es aquel que crece más rápido.

\subsection{Notación $O$}

Empezaremos con el concepto intuitivo de que una función domina asintóticamente a otra según la entrada crece. Para ello damos la siguiente definición.

\begin{definicion}
	Sean $f$ y $g$ dos funciones definidas en $\N$, y cuyas imágenes pertenecen a $\R^+$. Diremos que $f$ es de orden $g$, notado como $O(g(n))$, si, y solo si, $\exists k \in \N$ y $\exists C \in \R^+$ tales que se cumple lo siguiente:
	
	$$f(n) \leq Cg(n) \;\;\;\forall n \in \N;\; n \geq k$$
\end{definicion}

Esta definición nos dice que una función domina a otra dada si la primera multiplicada por una constante es mayor que la segunda para toda entrada a partir de cierto punto. Veamos ahora algunos ejemplos:

\begin{ejemplo}
	Probar que $f(n) = 3n^2 + 1$ es $O(n^2)$.\\
	
	Tomando $k=1$ y $C=4$, podemos ver fácilmente usando inducción sobre $n$ que $3n^2 + 1 \leq 4n^2\;\forall n \geq 1$, luego podemos asegurar que $3n^2 + 1 = O(n^2)$.\\
	
	\begin{itemize}
		\item Si $n=1$, entonces $3 \cdot 1^2 + 1 = 4 \leq 4$, luego se cumple el caso inicial.
		\item Suponiendo cierto para $n$, comprobemos para $n + 1$. Entonces $3(n+1)^2 + 1 = 3n^2 + 6^n + 3 + 1 \leq 4n^2 + 6n + 3 \leq 4n^2 + 8n + 4 = 4(n+1)^2$, luego hemos probado lo que queríamos.
	\end{itemize}
\end{ejemplo}

\subsection{Notación $\Omega$}

Intuitivamente, el concepto de la notación $\Omega$ es el opuesto al concepto de la notación $O$. Lo vemos más rápido en la definición.

\begin{definicion}
	Sean $f$ y $g$ dos funciones definidas en $\N$, y cuyas imágenes pertenecen a $\R^+$. Diremos que $f$ es de orden $g$, notado como $\Omega(g(n))$, si, y solo si, $\exists k \in \N$ y $\exists C \in \R^+$ tales que se cumple lo siguiente:
	
	$$f(n) \geq Cg(n) \;\;\;\forall n \in \N;\; n \geq k$$
\end{definicion}

Viendo la definición, es inmediato ver que, dadas dos funciones $f,g:\N \to \R^+$, entonces $f(n) = O(g(n)) \Leftrightarrow f(n) = \Omega(g(n))$. Algunos ejemplos son:

\begin{ejemplo}
	$3^n = \Omega(2^n)$
\end{ejemplo}

\begin{ejemplo}
	$n^3 + 2n + 3 \neq \Omega(n^4)$
\end{ejemplo}

Realmente este concepto es exactamente igual que el anterior, solo que la acotación la hacemos por debajo en vez de por arriba. Pasaremos entonces al concepto siquiente.

\subsection{Notación $\Theta$}

Este concepto no es más que una manera de indicar que dos funciones se acotan asintóticamente, o lo que es lo mismo, que crecen con la misma rapidez. También se le conoce como el ``orden exacto''. Para ser más exactos, esta es la definición.

\begin{definicion}
	Sean $f$ y $g$ dos funciones definidas en $\N$, y cuyas imágenes pertenecen a $\R^+$. Diremos que $f$ es de orden exacto $g$, notado como $\Theta(g(n))$, si, y solo si
	
	$$f(n) = O(g(n))\;\wedge\;f(n) = \Omega(g(n))$$
\end{definicion}

\subsection{Notación $O^\sim$}

Algunas veces es complicado calcular la complejidad exacta, y puede que nos baste simplemente probar que nuestro algoritmo pertenece a una clase que sigue siendo polinómica. Por ello hacemos la siguiente definición:

\begin{definicion}
	Sea $f:\N \to \R^+$ y definimos $O^\sim(f(n)) = O(f(n) \cdot poly(\log(f(n)))$, donde $poly(n)$ es una función polinómica en $n$.
\end{definicion}

Con esta definición, tenemos que $O^\sim(\log^k(n)) = O(\log^k(n) \cdot poly(\log(\log^k(n))) = O(\log^{k+\epsilon}(n))$.\\

Consideramos $\log(n)$ como el logaritmo en base $2$, y $ln$ como el logaritmo natural.

\endinput
