% !TeX root = ../libro.tex
% !TeX encoding = utf8
\chapter{Conclusiones y Vías Futuras}

Una vez demostrada la validez del algoritmo, vamos a dar algunas conclusiones sobre la demostración, así como algunas mejoras que podrían mejorar aún más el algoritmo.

\section{Conclusiones}

Al final hemos conseguido uno de los objetivos principales del trabajo: demostrar que el algoritmo propuesto determina que un número es primo si, y solo si, dicho número es primo. El análisis de su complejidad lo realizaremos en la siguiente parte del trabajo.\\

Una de las cosas más destacables de la demostración de la validez del algoritmo \textbf{AKS} es el uso de herramientas elementales de matemáticas.\\

En las primeras versiones del trabajo sobre el algoritmo \textbf{AKS} \cite{primes_is_in_p}, los autores hacían uso de herramientas más avanzadas de \textit{Teoría de Números} para poder realizar las acotaciones de \eqref{grupo_polinomios_introspectivos}.\\

Gracias a la introducción de \eqref{grupo_introspectivos} y usando propiedades de las raíces de los polinomios ciclotómicos en un cuerpo, se puede evitar el uso de herramientas no elementales para la demostración del algoritmo.\\

Esto resulta en una demostración elegante y al alcance de la mayoría de los matemáticos.

\section{Mejoras del Algoritmo AKS}

Una vez probada la validez de nuestro algoritmo, es en la siguiente parte donde comprobaremos que dicho algoritmo tiene complejidad polinómica. En específico, veremos que usando buenos algoritmos de multiplicación y división de números enteros y polinomios, la complejidad es $O^\sim(\log{21/2}(n))$.\\

Esta eficiencia se deduce de que $r = O(\log^5(n))$ por \autoref{cota_superior_r_log5} y que la cantidad de iteraciones a realizar en el paso $5$ es $\lfloor \sqrt{\phi(r)}\log(n) \rfloor$.\\

Mejorando la cota para $r$ o reduciendo la cantidad de iteraciones del paso $5$, podremos asegurar una mejor complejidad. Vamos a discutir ambos casos.

\subsection{Cota de $r$}

Como ya demostramos en \autoref{cota_superior_r_log5}, podemos encontrar $r = O(\log^5(n))$. En el mejor de los casos, y dado que $ord_r(n) > \log^2(n)$, podremos encontrar un $r = O(\log^2(n))$, lo cual reduce significativamente la complejidad del algoritmo. Existen dos conjeturas de las cuales se puede deducir dicha afirmación.

\begin{conjetura}{(\textbf{Conjetura de Artin}).}\label{conjetura_de_artin}
	Sea $n \in \N$ de manera que no es una potencia perfecta. Entonces, la cantidad de primos $q$ tales $q \leq m$ para algún $m > 0$ y tales que $ord_q(n) = q-1$ es asintóticamente $A(n)\frac{m}{\ln(n)}$, donde $A(n) > 0.35$ es la constante de Artin.
\end{conjetura}

Esta conjetura, en caso de que se cumpliera para un cierto $m = O(\log^2(n))$, implicaría la existencia de un $r = O(\log^2(n))$ cumpliendo las condiciones necesarias. Esta conjetura también es cierta en caso de que la \textit{Hipótesis Generalizada de Riemann} \ref{hipotesis_generalizada_de_riemann} sea cierta.\\

Para asegurar dicho $m$, enunciamos esta segunda conjetura.

\begin{conjetura}{(\textbf{Conjetura de la Densidad de Sophie-Germain}).}\label{conjetura_de_sophie_germain}
	La cantidad de primos $q$ con $q \leq m$ para algún $m > 0$ tales que $2q + 1$ también es primo es asintóticamente $\frac{2C_2m}{\ln^2(m)}$. $C_2 \simeq 0.66$ es la constante de los números primos gemelos.\\
	
	Los números primos con esta propiedad se les suele conocer como \textit{Primos de Sophie-Germain}.
\end{conjetura}

Si esta segunda conjetura fuera cierta, entonces podemos asegurar que existe un $m = O(\log^2(n))$ que cumple la conjetura de Artin, luego concluiríamos que existe un $r = O^\sim(\log^2(n))$ tal que $ord_r(n) > \log^2(n)$, como queríamos.\\

A pesar de que se sigue realizando trabajo en demostrar estas dos conjeturas, una versión más débil fue demostrada por el matemático \textit{Fouvry} en \cite{fouvry_1985}.

\begin{lema}
	Existen $c, n_0$ con $c > 0$ tales que, para todo $x \geq n_0$, se tiene
	
	\begin{equation}
	|\left\{ q \mid p\text{ es primo},q \leq x\text{ y } P(q-1) > q^{2/3} \right\}| \geq c\frac{x}{\ln(x)}
	\end{equation}
\end{lema}

Usando este lema, se puede mejorar la cota de $r$.

\begin{teorema}
	Existe $r = O(\log^3(n))$ tal que $ord_r(n) > \log^2(n)$.
\end{teorema}

\subsection{Iteraciones del Paso $5$}

Otra manera de optimizar aún más el algoritmo es reduciendo la cantidad de iteraciones que hay que realizar en el paso $5$, donde la implementación actual realiza $\lfloor \sqrt{\phi(r)}\log(n) \rfloor$ iteraciones.\\

La razón de estas iteraciones se basa en que necesitamos asegurar que el tamaño del grupo $\mathcal{G}$ sea suficientemente grande. Si podemos encontrar alguna manera de demostrar que un $\mathcal{G}$ puede estar generado por menos polinomios del estilo $X + a$, la cantidad de iteraciones también se reduciría, lo cual mejoraría la eficiencia del algoritmo.

\subsection{Otras mejoras}

Se puede mejorar aún más la complejidad del algoritmo hasta $O^\sim(\log^3(n))$ si la siguiente conjetura es cierta.

\begin{conjetura}\label{conjetura_aks}
	Sea $r$ un número primo que no divide a $n$ tal que se cumple
	
	\begin{equation}
	(X - 1)^n \equiv X^n - 1 \mod(n, X^r - 1)
	\end{equation}
	
	Entonce $n$ es primo o $n^2 \equiv 1 \mod(r)$.
\end{conjetura}

Esta congruencia se puede determinar en tiempo $O^\sim(r\log^2(n))$, y como dicho $r$ podemos encontrarlo en el intervalo $[2, 4\log(n)]$, concluimos con el análisis.\\

Existen argumentos heurísticos que afirman que dicha conjetura probablemente no sea cierta. Sin embargo, algunas modificaciones de la conjetura, como exigir que $r > \log(n)$, pueden seguir siendo ciertas.

\endinput
